\documentclass{beamer}

\usetheme{Madrid}
\usecolortheme{default}

\usepackage{graphicx}
\usepackage{hyperref}
\usepackage{multicol}

\title{VAST Challenge 2022 \\ Challenge 2}
\author[Arrigo M., Galardi G., Noventa N.]{
Matteo Arrigo \and
Giuseppe Galardi \and
Nicola Noventa \\[0.5em]
}
\date{}

\begin{document}

%------------------------------------------------


\begin{frame}
  \titlepage
\end{frame}


%------------------------------------------------

\begin{frame}{Challenge 2: Patterns of Life}

    % High-level summary
    \textbf{Objective:} Use visual analytics to characterize daily routines, travel patterns, and temporal changes within the fictitious city of Engagement.

    \vspace{0.5cm}

    \begin{block}{Key Analytical Tasks}
        \begin{itemize}
            \setlength\itemsep{1em} % Adds breathing room between items
            
            \item \textbf{City Characterization:} 
            Identify distinct areas of the city based on volunteer demographics and data.
            
            \item \textbf{Traffic \& Bottlenecks:} 
            Locate the busiest areas and identify potential traffic hazards.
            
            \item \textbf{Daily Routines:} 
            Analyze and contrast the daily patterns of two representative participants.
            
            \item \textbf{Temporal Changes:} 
            Examine how city patterns evolve over time and across seasons.
        \end{itemize}
    \end{block}

\end{frame}
%------------------------------------------------

\begin{frame}{Infrastructure}
  \includegraphics[width=1.0\linewidth]{presentation/diagram.jpeg}
\end{frame}

%------------------------------------------------

\begin{frame}{Buildings Map}

\vspace{-1em}
\begin{columns}[T]
    %---------------- LEFT COLUMN ----------------
    \begin{column}{0.75\textwidth}
        \centering
        \vspace{-1em}
        \begin{figure}
            \centering
            \includegraphics[width=0.85\linewidth]{presentation/buildings_map.png}
            % \caption{Interactive map showing all buildings in Engagement city categorized by venue type}
            \label{fig:buildings_map}
        \end{figure}
    \end{column}

    %---------------- RIGHT COLUMN ----------------
    \begin{column}{0.3\textwidth}
        \vspace{3em}
        \raggedright
        \textbf{Purpose:} Provides a spatial baseline for understanding the city's structure and serves as reference context for all subsequent visualizations.
        
    \end{column}
\end{columns}
\end{frame}

\begin{frame}{Parallel Coordinates Plot}

%---------------- TOP ROW ----------------
\begin{columns}[T]
    \begin{column}{0.5\textwidth}
        \vspace{2em}
        \centering
        \includegraphics[width=\linewidth]{presentation/pcp.png}

        \textbf{Purpose:} Baseline for multidimensional analysis across participant activity patterns
    \end{column}

    % RIGHT: CLASS EXPLANATION
    \begin{column}{0.5\textwidth}
        \textbf{Five dimensions}
        \begin{itemize}
            \item \textbf{Work}: work-related activities, including workplace visits and work–home commuting
            \item \textbf{Home}: time spent at home, based on apartment venue visits
            \item \textbf{Social}: social and recreational activities, including pub visits and recreation-related travel
            \item \textbf{Food}: food-related activities, including restaurant visits and eating-purpose travel
            \item \textbf{Travel}: total mobility, aggregating all recorded travel activities
        \end{itemize}
    \end{column}
\end{columns}

\vspace{1.5em}

\end{frame}



%================================================
% QUESTION 1
%================================================

\begin{frame}
\begin{center}
    {\Large \textbf{Question 1}}
    
    \vspace{1.5em}
    
    {\large Assuming the volunteers are representative of the city's population, \textbf{characterize the distinct areas} of the city that you identify.}
\end{center}
\end{frame}



\begin{frame}{Area Characteristics Visualization}

% \vspace{-1em}
\begin{columns}[T]
    %---------------- LEFT COLUMN ----------------
    \begin{column}{0.5\textwidth}
        \centering
        \includegraphics[width=1.1\linewidth]{presentation/area_characteristics.png}
        % \caption{Heatmap displaying population density and activity patterns across different areas}
        \label{fig:area_characteristics}
    \end{column}

    %---------------- RIGHT COLUMN ----------------
    \begin{column}{0.5\textwidth}
        % \textbf{Why useful for Q1:}
        \vspace{3em}
        \begin{itemize}
            \item Reveals spatial distribution patterns of population and activity
            % \item Identifies distinct areas with different characteristics (residential, commercial, recreational)
            \item Shows intensity variations through color-coded heatmap visualization
            \item Enables comparison of different metrics (population, activity) across grid cells
        \end{itemize}
    \end{column}
\end{columns}

\begin{figure}
    \centering
    \includegraphics[width=0.6\linewidth]{presentation/area_characteristics.png}
    \caption{Heatmap displaying population density and activity patterns across different areas}
    \label{fig:area_characteristics}
\end{figure}

\end{frame}

\begin{frame}{Example: Population and Housing Patterns}
\vspace{-1em}
\begin{columns}[T]
    %---------------- LEFT COLUMN ----------------
    \begin{column}{0.5\textwidth}
        \centering
        \includegraphics[height=0.4\textheight]{report/1/population.png}\\
        \footnotesize Fig.: Population Density

        \vspace{1em}
        \begin{itemize}\footnotesize
            \item Correlation between high population density areas and apartment building concentration.
        \end{itemize}
    \end{column}

    %---------------- RIGHT COLUMN ----------------
    \begin{column}{0.5\textwidth}
    
        \centering
        \includegraphics[height=0.4\textheight]{report/1/house_size.png}\\
        \footnotesize Fig.: Average Household Size

        \vspace{1em}

        \begin{itemize}\footnotesize
            \item Larger homes correspond to lower density, while smaller homes cluster in denser areas.
        \end{itemize}
    \end{column}
\end{columns}
\end{frame}


%================================================
% QUESTION 2
%================================================

\begin{frame}
\begin{center}
    {\Large \textbf{Question 2}}
    
    \vspace{1.5em}
    
    {\large Where are the \textbf{busiest areas} in Engagement?}
    
    \vspace{0.5em}
    
    {\large Are there \textbf{traffic bottlenecks} that should be addressed?}
\end{center}
\end{frame}

\begin{frame}{Traffic Patterns Visualization}

\begin{columns}[T]
    %---------------- LEFT COLUMN ----------------
    \begin{column}{0.5\textwidth}
        \centering
        \includegraphics[width=1.1\linewidth]{presentation/traffic_patterns.png}
        \label{fig:traffic_patterns}
    \end{column}

    %---------------- RIGHT COLUMN ----------------
    \begin{column}{0.5\textwidth}
        \vspace{4em}
        \begin{itemize}
            \item Displays aggregated visit counts per location using bubble size and color intensity
            \item Directly identifies the busiest areas through visual emphasis
            \item Enables filtering by time period and day type for pattern analysis
            % \item Provides quantitative metrics (total visits, busiest location)
        \end{itemize}
    \end{column}
\end{columns}

\end{frame}

\begin{frame}{Example: Weekday vs Weekend Traffic}
    \begin{multicols}{2}
    \includegraphics[width=0.9\linewidth]{report/2/traffic_weekdays_morning.png}
    \includegraphics[width=0.9\linewidth]{report/2/traffic_weekends_morning.png}
    
    \end{multicols}
    Weekends feature traffic clustering around restaurants and pubs, while weekdays display a more evenly distributed, commute-driven pattern.
\end{frame}


\begin{frame}{Traffic Density Visualization}

\begin{columns}[T]
    %---------------- LEFT COLUMN ----------------
    \begin{column}{0.5\textwidth}
        \centering
        \includegraphics[width=1.1\linewidth]{presentation/traffic_density.png}
        \label{fig:traffic_density}
    \end{column}

    %---------------- RIGHT COLUMN ----------------
    \begin{column}{0.5\textwidth}
        \vspace{1em}
        \begin{itemize}
            \item Visualizes movement flows between locations over time
            \item Reveals primary connectivity patterns
            \item Shows temporal dynamics with playback controls
            \item Identifies potential bottlenecks through edge density

        \end{itemize}
    \end{column}
\end{columns}

\end{frame}

\begin{frame}{Esampe: Afternoon Convergence}
\begin{columns}
    \column{0.5\textwidth}
    \centering
    \includegraphics[width=0.85\linewidth]{report/2/flow_density_weekends_2pm.png}\\
    \footnotesize Fig.: Weekend afternoon flows (strong clustering)

    \column{0.5\textwidth}
    \centering
    \includegraphics[width=0.85\linewidth]{report/2/flow_density_all_1pm.png}\\
    \footnotesize Fig.: All-days 1 PM flows (persistent attractors)
\end{columns}
\end{frame}


%================================================
% QUESTION 3
%================================================

\begin{frame}
\begin{center}
    {\Large \textbf{Question 3}}
    
    \vspace{1.5em}
    
    {\large Participants have given permission to have their daily routines captured.}
    
    \vspace{0.5em}
    
    {\large Choose \textbf{two different participants} with different routines and describe their \textbf{daily patterns}, with supporting evidence.}
\end{center}
\end{frame}

\begin{frame}{Daily Routines: Timeline Visualization}

%---------------- TOP: PLOT ----------------
\centering
\includegraphics[width=0.70\linewidth]{presentation/daily_routines_1.png}

% \vspace{1em}

%---------------- BOTTOM: TEXT IN TWO COLUMNS ----------------
\begin{columns}[T]
    \begin{column}{0.5\textwidth}
        \begin{itemize}
            \item Displays hour-by-hour activity breakdown for each participant
            \item Color-coded activity categories make routine patterns immediately visible
        \end{itemize}
    \end{column}

    \begin{column}{0.5\textwidth}
        \begin{itemize}
            \item Supports direct comparison of daily routines between two participants
            \item Time-based filtering enables analysis of specific periods or day types
        \end{itemize}
    \end{column}
\end{columns}

\end{frame}


\begin{frame}{Daily Routines: Travel Routes Visualization}

\begin{columns}[T]
    %---------------- LEFT COLUMN ----------------
    \begin{column}{0.5\textwidth}
        \centering
        \includegraphics[width=1.1\linewidth]{presentation/daily_routines_2.png}
        \label{fig:daily_routines_routes}
    \end{column}

    %---------------- RIGHT COLUMN ----------------
    \begin{column}{0.5\textwidth}
        \vspace{2em}
        \begin{itemize}
            \item Maps actual geographic movements and key locations (home, work)
            \item Shows spatial patterns and travel distances
            \item Reveals routine differences through distinct route networks
            \item Activity breakdown summary provides quantitative support
        \end{itemize}
    \end{column}
\end{columns}
\end{frame}


%================================================
% QUESTION 4
%================================================

\begin{frame}
\begin{center}
    {\Large \textbf{Question 4}}
    
    \vspace{1.5em}
    
    {\large Over the span of the dataset, how do \textbf{patterns change} in the city?}
\end{center}
\end{frame}


\begin{frame}{Temporal Analysis: Activity Calendar}

\begin{columns}[T]
    %---------------- LEFT COLUMN ----------------
    \begin{column}{0.5\textwidth}
        \centering
        \includegraphics[width=1.1\linewidth]{presentation/temporal_analysis_calendar.png}
        \label{fig:temporal_calendar}
    \end{column}

    %---------------- RIGHT COLUMN ----------------
    \begin{column}{0.5\textwidth}
        \vspace{1em}
        \begin{itemize}
            \item Calendar heatmap shows activity intensity for every day in the dataset
            \item Geographic filtering enables location-specific temporal analysis
            \item Timeline chart reveals long-term trends and seasonal variations
            \item Activity type filtering identifies how different venue categories change over time
        \end{itemize}
    \end{column}
\end{columns}
\end{frame}

%------------------------------------------------
\begin{frame}{Example: Residential vs Working Areas}
\begin{multicols}{2}
\centering
\includegraphics[width=\linewidth]{report/4/working_area_morning.png}\\
\footnotesize Fig.1: Working area – morning activity

\includegraphics[width=\linewidth]{report/4/working_area_evening.png}\\
\footnotesize Fig.2: Working area – evening activity

\columnbreak

\centering
\includegraphics[width=\linewidth]{report/4/residential_area_morning.png}\\
\footnotesize Fig.3: Residential area – morning activity

\includegraphics[width=\linewidth]{report/4/residential_area_evening.png}\\
\footnotesize Fig.4: Residential area – evening activity
\end{multicols}
\end{frame}

\begin{frame}{Seasonal Comparison: Radar Chart}

\begin{columns}[T]
    %---------------- LEFT COLUMN ----------------
    \begin{column}{0.5\textwidth}
        \centering
        \includegraphics[width=1.1\linewidth]{presentation/temporal_analysis_diamond.png}
        \label{fig:seasonal_radar}
    \end{column}

    %---------------- RIGHT COLUMN ----------------
    \begin{column}{0.5\textwidth}
        \vspace{3em}
        \begin{itemize}
            % \item Compares activity patterns across four seasons (Spring, Summer, Fall, Winter)
            \item Multi-dimensional view shows how different activity types change seasonally
            \item Overlapping polygons enable direct visual comparison between seasons
            \item Reveals seasonal preferences and behavioral shifts in the city
        \end{itemize}
    \end{column}
\end{columns}
\end{frame}

\begin{frame}{Venue Visits Over Time}

%---------------- TOP: PLOT ----------------
\centering
\includegraphics[width=0.7\linewidth]{presentation/venue_visits.png}

% \vspace{1em}

%---------------- BOTTOM: TEXT IN TWO COLUMNS ----------------
\begin{columns}[T]
    \begin{column}{0.5\textwidth}
        \begin{itemize}
            \item Tracks visit patterns for specific venues across the full dataset timespan
            \item Multiple time granularities (daily, weekly, monthly) reveal trends at different scales
        \end{itemize}
    \end{column}

    \begin{column}{0.5\textwidth}
        \begin{itemize}
            \item Displays both total visits and unique visitors to distinguish recurring from new traffic
            \item Summary statistics highlight peak periods and changes in venue popularity over time
        \end{itemize}
    \end{column}
\end{columns}

\end{frame}

%================================================
% CONCLUSIONS
%================================================

\begin{frame}{Conclusions: Strengths}
\begin{itemize}
    \item The most effective visualizations are those addressing \textbf{bottlenecks} (Question 2). These charts clearly highlight \textbf{actual bottlenecks} within the city, and the applied \textbf{filters} make it possible to observe how they \textbf{evolve throughout the day}.
    \item The \textbf{comparison between two participants} and their routes (Question 3) is also a strong point. As discussed in the report, this visualization enables an \textbf{effective comparison} between participants with \textbf{different profiles and daily routines}.
    \item The \textbf{heatmap} provides a \textbf{clear and intuitive} way to highlight \textbf{areas of interest} across the city.
\end{itemize}
\end{frame}

\begin{frame}{Conclusions: Limitations}
\begin{itemize}
    \item We experimented with several visualizations to identify \textbf{seasonal or monthly patterns}. For example, we expected a \textbf{decrease in work-related activities} during the \textbf{summer period}, but our analysis did not reveal such trends.
    \item To improve \textbf{efficiency}, some visualizations rely on \textbf{aggregated data tables}. While beneficial for performance, this choice may \textbf{limit the discovery of fine-grained temporal patterns}.
    \item Due to \textbf{time constraints}, we were not able to perform \textbf{advanced statistical analyses} on the data (e.g., \textbf{clustering}, \textbf{dimensionality reduction}).
\end{itemize}
\end{frame}



%================================================
% WORK DIVISION
%================================================

\begin{frame}{Work Division}
\begin{columns}[T]
    \begin{column}{0.33\textwidth}
        \centering
        \textbf{Matteo Arrigo}
        
        \vspace{1em}
        
        \begin{itemize}\small
            \item Frontend
            \item Question 1, 2
            \item Project Presentation
            \item Project Report
        \end{itemize}
    \end{column}
    
    \begin{column}{0.33\textwidth}
        \centering
        \textbf{Giuseppe Galardi}
        
        \vspace{1em}
        
        \begin{itemize}\small
            \item Backend
            \item Infrastructure and Setup
            \item Video Presentation
        \end{itemize}
    \end{column}
    
    \begin{column}{0.33\textwidth}
        \centering
        \textbf{Nicola Noventa}
        
        \vspace{1em}
        
        \begin{itemize}\small
            \item Frontend
            \item Question 3, 4
            \item Project Presentation
            \item Project Report
        \end{itemize}
    \end{column}
\end{columns}

\end{frame}

%------------------------------------------------
\begin{frame}
    \centering
    \Huge \textbf{Thank You}

    \vspace{1cm}
    \large Questions?
\end{frame}

\end{document}
